% Article.tex
% Sam Harrison 2018
% MIT License: a full version of the license is included in the LICENSE file.

% For TeXShop, TeXWorks, etc
% !TEX TS-program = LuaLaTeX
% !TEX encoding = UTF-8 Unicode
% !TEX spellcheck = en-GB

%%%%%%%%%%%%%%%%%%%%%%%%%%%%%%%%%%%%%%%%%%%%%%%%%%%%%%%%%%%%%%%%%%%%%%%%%%%%%%%
% Preamble

%\documentclass[a4paper, 10pt, twocolumn]{lt_article}
\documentclass[a4paper, 11pt]{lt_article}

\usepackage{lt_basics}
\usepackage{lt_maths}
\usepackage{lt_utilities}

% For maths examples
\DeclareMathOperator\Res{Res}
\newcommand*\diff{\mathop{}\!\mathup{d}}

\usepackage{amsthm}
\newtheorem{theorem}{Theorem}

\title{Template for the \texttt{lt\_article} class}
\author{A.~N.~Author, B.~N.~Author \& C.~N.~Author}
%\author{
%    A.~N.~Author\thanks{Institution A}\thanksgap{0.5ex},
%    B.~N.~Author\thanks{Institution B}\thanksgap{0.5ex} \&
%    C.~N.~Author\thanksmark{1}\thanksgap{0.75ex}\thanksmark{2}}
\date{\today}
\header{Article Template}

%%%%%%%%%%%%%%%%%%%%%%%%%%%%%%%%%%%%%%%%%%%%%%%%%%%%%%%%%%%%%%%%%%%%%%%%%%%%%%%

\begin{document}

% Generate the title
\maketitle

% % % % % % % % % % % % % % % % % % % % % % % % % % % % % % % % % % % % % % %
\begin{abstract} % {{{

\noindent\blindtext

\end{abstract}
% }}}
% % % % % % % % % % % % % % % % % % % % % % % % % % % % % % % % % % % % % % %
\section{Introduction} % {{{

\Blindtext[1]

% - - - - - - - - - - - - - - - - - - - - - - - - - - - - - - - - - - - - - - -
\subsection{Subsection} % {{{

\Blindtext[3]

% }}}
% - - - - - - - - - - - - - - - - - - - - - - - - - - - - - - - - - - - - - - -
\subsection{Mathematics} % {{{

\blindtext

% See https://tex.stackexchange.com/q/425098
\begin{theorem}[Residue theorem]
  Let $f$ be analytic in the region $G$ except for the isolated
  singularities $a_1,a_2,\dots,a_m$. If $\gamma$ is a closed
  rectifiable curve in $G$ which does not pass through any of the
  points $a_k$ and if $\gamma\approx 0$ in $G$, then
  \[
    \frac{1}{2\pi i} \int\limits_\gamma f\Bigl(x^{\mathbf{N}\in\mathbb{C}^{N\times 10}}\Bigr)
    = \sum_{k=1}^m n(\gamma;a_k)\Res(f;a_k)\,.
  \]
\end{theorem}

\begin{theorem}[Maximum modulus]
  Let $G$ be a bounded open set in $\BbbC$ and suppose that $f$ is a
  continuous function on $G^-$ which is analytic in $G$. Then
  \[
    \max\{\, |f(z)|:z\in G^- \,\} = \max\{\, |f(z)|:z\in \partial G \,\}\,.
  \]
\end{theorem}

First some large operators both in text:
$\iiint\limits_{Q}f(x,y,z) \diff x \diff y \diff z$
and
$\prod_{ \gamma \in \Gamma_{\bar{C}} }\partial(\tilde{X}_\gamma)$;\
and also on display
\[
  \iiiint\limits_{Q}f(w,x,y,z) \diff w \diff x \diff y \diff z
  \leq
  \oint_{\partial Q} f'\Biggl(\max\Biggl\{
  \frac{\Vert w\Vert}{\vert w^2+x^2\vert};
  \frac{\Vert z\Vert}{\vert y^2+z^2\vert};
  \frac{\Vert w\oplus z\Vert}{\vert x\oplus y\vert}
  \Biggr\}\Biggr)\,.
\]

\blindtext

% }}}
% - - - - - - - - - - - - - - - - - - - - - - - - - - - - - - - - - - - - - - -

% }}}
% % % % % % % % % % % % % % % % % % % % % % % % % % % % % % % % % % % % % % %
\section{Results} % {{{

% - - - - - - - - - - - - - - - - - - - - - - - - - - - - - - - - - - - - - - -
\subsection{Subsection} % {{{

\Blindtext[1]

\subsubsection{Subsubsection}
\Blindtext[2]

\subsubsection{Subsubsection}
\Blindtext[1]

% }}}
% - - - - - - - - - - - - - - - - - - - - - - - - - - - - - - - - - - - - - - -
\subsection{Subsection} % {{{

\Blindtext[2]

% }}}
% - - - - - - - - - - - - - - - - - - - - - - - - - - - - - - - - - - - - - - -

% }}}
% % % % % % % % % % % % % % % % % % % % % % % % % % % % % % % % % % % % % % %
\newpage
% % % % % % % % % % % % % % % % % % % % % % % % % % % % % % % % % % % % % % %
\section{Examples} % {{{

\textrm{\blindtext}

\textit{\blindtext}

\textbf{\blindtext}

\textsc{\blindtext}

\newpage

\textsf{\blindtext}

\textsf{\textit{\blindtext}}

\textsf{\textbf{\blindtext}}

\texttt{\blindtext}

\newpage

\begin{itemize}
%\begin{enumerate}[label=\Roman*., widest*=4]%, itemindent=\parindent]

    \item \uline{1234567\sout{890}} \num{12345} \SI{67.8956}{\um/\s^2}
        \SI{9999}{\hertz} \ang{90}

    \item \LuaTeX\ \LuaLaTeX

    \item \enquote{blah \enquote{blah}} blah

    \item \eg \ie \cf \etc

    \item Quoi?

\end{itemize}
%\end{enumerate}

% }}}
% % % % % % % % % % % % % % % % % % % % % % % % % % % % % % % % % % % % % % %

\end{document}

%%%%%%%%%%%%%%%%%%%%%%%%%%%%%%%%%%%%%%%%%%%%%%%%%%%%%%%%%%%%%%%%%%%%%%%%%%%%%%%
