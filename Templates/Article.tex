% Article.tex
% Sam Harrison 2018
% MIT License: a full version of the license is included in the LICENSE file.

% For TeXShop, TeXWorks, etc
% !TEX TS-program = LuaLaTeX
% !TEX encoding = UTF-8 Unicode
% !TEX spellcheck = en-GB

%%%%%%%%%%%%%%%%%%%%%%%%%%%%%%%%%%%%%%%%%%%%%%%%%%%%%%%%%%%%%%%%%%%%%%%%%%%%%%%
% Preamble {{{

%\documentclass[a4paper, 10pt, twocolumn]{lt_article}
\documentclass[a4paper, 11pt]{lt_article}

\usepackage{lt_basics}
\usepackage{lt_bib}
\usepackage{lt_floats}
\usepackage{lt_maths}
\usepackage{lt_utilities}
\usepackage{lt_hyperref}
%\usepackage[print]{lt_hyperref}

\title{Template for the \texttt{lt\_article} class}
%\author{A.~N.~Author, B.~N.~Author \& C.~N.~Author}
\author[a,\corrmark]{A.~N.~Author}
\author[a,b]{B.~N.~Author}
\author[c]{C.~N.~Author}
\affiliation{Institution 1}
\affiliation{Institution 2}
\affiliation{Institution 3}
\corresponding{%
    Corresponding author:
    \hiddenurl{mailto:}{a.author@institution.org}{}
}
\date{\today}
\header{Article Template (v0.2.0)}

\addbibresource{PlaceholderBibliography.bib} % \string~

% }}}
%%%%%%%%%%%%%%%%%%%%%%%%%%%%%%%%%%%%%%%%%%%%%%%%%%%%%%%%%%%%%%%%%%%%%%%%%%%%%%%

\begin{document}

% Generate the title
\maketitle
\printAffiliations

%%%%%%%%%%%%%%%%%%%%%%%%%%%%%%%%%%%%%%%%%%%%%%%%%%%%%%%%%%%%%%%%%%%%%%%%%%%%%%%
\begin{abstract} % {{{

\blindtext

\end{abstract} % }}}
%%%%%%%%%%%%%%%%%%%%%%%%%%%%%%%%%%%%%%%%%%%%%%%%%%%%%%%%%%%%%%%%%%%%%%%%%%%%%%%
% ToC {{{

% Disable microtype locally so table entries line up properly
%\microtypesetup{protrusion=false}
%\tableofcontents
%\microtypesetup{protrusion=true}

% }}}
%%%%%%%%%%%%%%%%%%%%%%%%%%%%%%%%%%%%%%%%%%%%%%%%%%%%%%%%%%%%%%%%%%%%%%%%%%%%%%%
\section{Introduction} % {{{

\blindtext
Which should makes things clear\footnote{Test}.
See any of \textcite{Bayes}, the online resources at
\hiddenurl{https://}{ctan.org}{/pkg/hyperref}, or the examples in
\Ref{Section}{sec:Examples} for more information \autocite{GravitationalWaves}.

%------------------------------------------------------------------------------
\subsection{Subsection} % {{{

\Blindtext[3]

% }}}
%------------------------------------------------------------------------------
\subsection{Mathematics} % {{{

\begin{equation}
\begin{split}
    \func{\FreeEnergy}{D, \lambda}
    & = \expecWRT{ \lnFunc{\cDist{p}{D}{\theta}} }{ \pDist{q}{\theta}{\lambda} }
    - \DKL{ \pDist{q}{\theta}{\lambda} }{ \dist{p}{\theta} }
    \\
    & = \expecWRT{ \lnFunc{\dist{p}{D, \theta}} }{ \pDist{q}{\theta}{\lambda} }
    - \expecWRT{ \lnFunc{\pDist{q}{\theta}{\lambda}} }{ \pDist{q}{\theta}{\lambda} }
    \\[5pt]
    \hat{\lambda}
    & = \Func{ \argmax_{\lambda} }{ \func{\FreeEnergy}{D, \lambda} }
\end{split}
\end{equation}

\begin{equation}
    \del_{\lambda} \func{\FreeEnergy}{D, \lambda}
    \approx
    \frac{1}{N} \sum_{n=1}^{N} \Bigl(
        \lnFunc{\dist{p}{D, \Theta_{n}}}
        - \lnFunc{\pDist{q}{\Theta_{n}}{\lambda}}
    \Bigr)
    \del_{\lambda} \lnFunc{\pDist{q}{\Theta_{n}}{\lambda}};
    \,\, \Theta_{n} \sim \pDist{q}{\theta}{\lambda}
\end{equation}

\blindtext

% }}}
%------------------------------------------------------------------------------

% }}}
%%%%%%%%%%%%%%%%%%%%%%%%%%%%%%%%%%%%%%%%%%%%%%%%%%%%%%%%%%%%%%%%%%%%%%%%%%%%%%%
\section{Results} % {{{

%------------------------------------------------------------------------------
\subsection{Subsection} % {{{

\Blindtext[1]

\subsubsection{Subsubsection}
\blindtext

% - - - - - - - - - - - - - - - - - - - - - - - - - - - - - - - - - - - - - - -
\begin{figure*}[tbp] % {{{
\centering

\subcaptionbox{
    A grey square.
    \label{subfig:Square}
}{
    \includegraphics[width=0.45\linewidth]{PlaceholderImage.png}
}
%
\quad  % Force a bit of horizontal whitespace
%
\subcaptionbox{
    A grey rectangle, with a longer caption to illustrate text wrapping.
    \label{subfig:Rectangle}
}{
    \includegraphics[
        trim={0 0 0 {.1\height}}, clip, width=0.45\linewidth
    ]{PlaceholderImage.png}
}
%%
%\par\bigskip  % Force a bit of vertical whitespace
%%
%\subcaptionbox{
%    And a long grey rectangle, to make it look a bit like a face.
%    \label{subfig:LongRectangle}
%}{
%    \includegraphics[
%        trim={0 0 0 {.8\height}}, clip, width=0.8\linewidth
%    ]{PlaceholderImage.png}
%}

\caption{
    Some artistic shapes. Another long caption that is just designed to
    illustrate the different margin options. Stop reading! There's really
    nothing else coming.
}
\label{fig:Shapes}
\end{figure*} % }}}
% - - - - - - - - - - - - - - - - - - - - - - - - - - - - - - - - - - - - - - -

\blindtext
See \Ref{Figure}{fig:Shapes}, specifically \SubRef{Panel}{subfig:Rectangle}.

\subsubsection{Subsubsection}
\Blindtext[1]

% }}}
%------------------------------------------------------------------------------
\subsection{Subsection} % {{{

\Blindtext[2]

% }}}
%------------------------------------------------------------------------------

% }}}
%%%%%%%%%%%%%%%%%%%%%%%%%%%%%%%%%%%%%%%%%%%%%%%%%%%%%%%%%%%%%%%%%%%%%%%%%%%%%%%
% Bibliography {{{

\newrefcontext[sorting=nyt]
\printbibliography
\newrefcontext[sorting=ynt]

% }}}
%%%%%%%%%%%%%%%%%%%%%%%%%%%%%%%%%%%%%%%%%%%%%%%%%%%%%%%%%%%%%%%%%%%%%%%%%%%%%%%
\newpage
%%%%%%%%%%%%%%%%%%%%%%%%%%%%%%%%%%%%%%%%%%%%%%%%%%%%%%%%%%%%%%%%%%%%%%%%%%%%%%%
\section{Examples} % {{{
\label{sec:Examples}

\textrm{\blindtext}

\textit{\blindtext}

\textbf{\blindtext}

\textsc{\blindtext}

\newpage

\textsf{\blindtext}

\textsf{\textit{\blindtext}}

\textsf{\textbf{\blindtext}}

\texttt{\blindtext}

\newpage

\begin{itemize}
%\begin{enumerate}[label=\Roman*., widest*=4]%, itemindent=\parindent]

    \item \uline{1234567\sout{890}} \num{12345} \SI{67.8956}{\um/\s^2}
        \SI{9999}{\hertz} \ang{90}

    \item \LuaTeX\ \LuaLaTeX

    \item \enquote{blah \enquote{blah}} blah

    \item \eg \ie \cf \etc

    \item Quoi? \textit{Journal}

    \item ff fi fl ffi fb fh fj fk ft Qu Th

    \item \textsf{ff fi fl ffi fb fh fj fk ft Qu Th}

\end{itemize}
%\end{enumerate}

% }}}
%%%%%%%%%%%%%%%%%%%%%%%%%%%%%%%%%%%%%%%%%%%%%%%%%%%%%%%%%%%%%%%%%%%%%%%%%%%%%%%

\end{document}

%%%%%%%%%%%%%%%%%%%%%%%%%%%%%%%%%%%%%%%%%%%%%%%%%%%%%%%%%%%%%%%%%%%%%%%%%%%%%%%
